\documentclass[letterpaper]{article}

% used for unnumbered equations
\usepackage{amsmath}
% used for enumerating problem parts by letter
\usepackage{enumerate}
% used for margins
\usepackage[letterpaper,left=1.25in,right=1.25in,top=1in,bottom=1in]{geometry}
% used for header and page numbers
\usepackage{fancyhdr}
% used for preventing paragraph indentation
\usepackage[parfill]{parskip}
% used for section formatting as problem
\usepackage[explicit]{titlesec}
% used for subsection indentation
\usepackage{changepage}

\pagestyle{fancy}
\fancyfoot{}
% header/footer settings
\lhead{Kevin Nash}
\chead{EECS 345 -- Written Exercise 1}
\rhead{\today}
\cfoot{\thepage}

\titleformat{\section}{\normalfont\large\bfseries}{}{0em}{#1\ \thesection}
\newcommand{\problem}{\section{Problem}}
\newenvironment{shift}
    {\adjustwidth{3em}{0pt}}
    {\endadjustwidth}

\begin{document}

\problem{}
\begin{tabular}{l l}
    \texttt{<C>}\quad$\rightarrow$\quad\texttt{<T> | <V> = <C>} & assignment (command)\\
    \texttt{<T>}\quad$\rightarrow$\quad\texttt{<O> | <T> ? <C> : <C> } & ternary\\
    \texttt{<O>}\quad$\rightarrow$\quad\texttt{<A> | <C> || <C>} & or\\
    \texttt{<A>}\quad$\rightarrow$\quad\texttt{<N> | <C> \&\& <C>} & and\\
    \texttt{<N>}\quad$\rightarrow$\quad\texttt{<P> | !<C>} & not\\
    \texttt{<P>}\quad$\rightarrow$\quad\texttt{<B> | (<C>)} & parentheses\\
    \texttt{<B>}\quad$\rightarrow$\quad\texttt{<V> | true | false} & boolean\\
    \texttt{<V>}\quad$\rightarrow$\quad\texttt{x | y | z} & value
\end{tabular}

\problem{}
\texttt{<start$_1$>.typetable := <stmt$_3$>.typetable ; <start$_3$>.typetable}\\
\texttt{<start$_1$>.inittable := <stmt$_3$>.inittable ; <start$_3$>.inittable}

\texttt{<start$_2$>.typetable := <stmt$_4$>.typetable}\\
\texttt{<start$_2$>.typetable := <stmt$_4$>.inittable}

\texttt{<stmt$_1$>.typetable := <declare$_2$>.typetable}\\
\texttt{<stmt$_1$>.inittable := <declare$_2$>.inittable}

\texttt{<stmt$_2$>.inittable := <assign$_2$>.inittable}

\texttt{<declare$_1$>.typebinding := (<var>, <type>)}\\
\texttt{<declare$_1$>.initialized := (<var>, false)}

\texttt{<type$_1$>.type := integer}

\texttt{<type$_2$>.type := double}

\texttt{<assign$_1$>.initialized :=}\\ \ldots\quad \texttt{(typetable(<var>) = expression$_3$.type) ?\ (<var>, true) :\ (<var>, false)}

\texttt{<expression$_1$>.type :=}\\ \ldots\quad \texttt{(<expression$_4$>.type = integer \&\& <expression$_5$>.type = integer) ?\ integer :\ double}

\texttt{<expression$_2$>.type := <value$_4$>.type}

\texttt{<value$_1$>.type := typetable(<var>)}

\texttt{<value$_2$>.type := integer}

\texttt{<value$_3$>.type := double}

\problem{}
\begin{enumerate}
\item Within \texttt{<assign$_1$>} the variable type should be checked against the expression type. Specifically, if the following evalutates \texttt{false}, raise an error:

\texttt{typetable(<var>) = <expression$_3$>.type}

\item Within \texttt{<assign$_1$>} the variable should be checked as declared before it can be used. Specifically, if the following evaluates \texttt{true}, raise an error:

\texttt{typetable(<var>) = error}

\item Within \texttt{<value$_1$>} the variable should be checked as initialized before it can otherwise be used. Specifically, if the following evaluates \texttt{false}, raise an error:

\texttt{if inittable(<var>) = true}
\end{enumerate}

%\vspace{-0.1 in}
\problem{}
\textbf{Inner loop invariant:} $A[i]\geq A[0..i-1]$
\begin{shift}
    \textbf{Initialization:} At the start of the loop, $i = 0$ and thus, since $A[0..i]$ contains one element, ${A[i]\geq A[0..i-1]}$

    \textbf{Maintenance:} With each loop, $A[i]$ is assigned the larger value of $A[i-1]$ and $A[i]$, so the loop invariant holds.

    \textbf{Termination:} When the loop terminates, $i=bound-1$. Therefore, due to the loop invariant, $A[i]\geq A[0..bound-1]$.
\end{shift}

\textbf{Outer loop invariant:} $A[bound-1..n-1]$ is sorted in non-descending order and is greater than $A[0..bound-1]$.
\begin{shift}
    \textbf{Initialization:} At the start of the loop, $bound=n$. Therefore the array $A[bound-1..n-1]$ is empty, so the loop invariant holds trivially.

    \textbf{Maintenance:} An overview of the inner loop invariant maintains that $A[i]\leq A[j], \text{for}\ i<j$, which holds that the current array is sorted in non-decreasing order.

    \textbf{Termination:} When the loop terminates, $bound=0$. Therefore, we can apply a bound of the entire length of $A$ to our loop invariant, which holds that $A[0..n-1]$ is sorted in non-decreasing order.
\end{shift}

\problem{}
\textbf{Defining \texttt{<assign>}}:
\begin{verbatim}
Mstate(<var> = <expression>, s) = {
    s1 = Mstate(<expression>, s)
    name = Mname(<var>, s)
    val = Mvalue(<expression>, s)
    if name is error
        return error
    else
        s1 = Remove(name, s)
        s1 = Add(name, val, s)
    return s1
}
\end{verbatim}
\textbf{Defining \texttt{<if>}}:
\begin{verbatim}
Mstate(if <condition> then <statement1> else <statement2>, s) = {
    s1 = Mstate(<condition>, s)
    cond = Mboolean(<condition>, s)
    if cond is true
        s1 = Mstate(<statement1>, s)
    if cond is false
        s1 = Mstate(<statement2>, s)
    return s1
}
\end{verbatim}
\textbf{Defining \texttt{<while>}}:
\begin{verbatim}
Mstate(while <condition> <loop body>, s) = {
    s1 = Mstate(<condition>, s)
    cond = Mboolean(<condition>, s)
    loop if cond is true
        s1 = Mstate(<loop body>, s)
        cond = Mboolean(<condition>, s)
    return s1
}
\end{verbatim}

\end{document}
